%
\chapter{Das Projekt}
\label{chap:beschreibung}

\section{Projekt Beschreibung}
Im Rahmen des Studiums ist ein mit 12 ECTS gewichtetes Teamprojekt vorgesehen.
Das Projekt IoT Sensor kombiniert drei selbst erstellte Platinen zu einem funktionierenden System.

\textbf{Das $\mu$ - Controller-Board }
\newline
Auf der Hauptplatine wird ein Mikroprozessor der MSP430FR Familie und drei PMOD Schnittstellen angebracht. Über die PMOD Stecker soll mit den anderen Boards kommuniziert werden.
\newline

\textbf{ USB-UART-Interface}
\newline
Die Kommunikationsplatine wird über USB mit dem PC verbunden. Diese wandelt das differenzielle USB-Signal, mithilfe des FT231X, in ein serielles UART-Signal um. Das umgewandelte Signal wird dann über einen PMOD Stecker via UART an das Main-Board weitergeleitet. Auf der Platine befindet sich auch eine galvanische Trennung, welche den Stromkreis des Computers von dem des Mikrocontrollers trennt. Hierfür ist der ADUM1401 verantwortlich, welcher das Signal induktiv in den anderen Stromkreis überträgt.
\newline

\textbf{ 7-Segmentanzeige }
\newline
Die Anzeige Platine stellt eine 7 Segmentanzeige zur Verfügung, welche die Darstellung von Messwerten übernehmen könnte. Hierzu dient der AS1108 welcher die Übersetzung einer Zahl auf die Sieben Segmentanzeige übernimmt. Die Zahlen werden über SPI vom Mainboard erhalten.

Mit der Sensorplatine (nicht teil des Projektes) wird ebenfalls über SPI kommuniziert und liefert Messwerte an das Main-Board. Diese kann bei Bedarf bereits Fertig gekauft werden.

\section{Arbeitsschritte}

\subsection{Bauteilsuche}
Nachdem die Vorgabe vollständig ist, kann mit der Bauteilsuche begonnen werden. Einige Teile waren auch bereits vorgegeben. Kriterien beim auswählen der Bauteile waren:

\begin{itemize}
	\item Funktion, - Angaben müssen erfüllt werden
	\item Verfügbarkeit - Teile müssen einzeln bestellbar und auch in Zukunft noch Angeboten werden
	\item Preis
\end{itemize}

\subsection{Platinen Design}
Nachdem alle Teile gefunden sind und auch alle Datenblätter vorliegen, kann mit dem Design der Platine begonnen werden.
hierzu wurde das Tool Pulsonix verwendet. Nachdem der Schaltplan entworfen war, konnte das Platinen-Layout erstellt werden. Die größte Hürde hierbei war das Konfigutieren des korrekten Rasters für Bauteile und Vias.
Zu beachten war auch die Positionierung der Bauteile. Manche Bauteile mussten möglichst nah an anderen Bauteilen angebracht werden. So beim FT231X, wo ein differentielles Signal anliegt.
Auch mussten an den PMOD Steckern zwei Kondensatoren angebracht werden um Störsignale welche beispielsweise beim Einstecken entstehen auf die Masse abzuleiten.

\subsection{Hardware bestücken}
Die Hardware wurde bei \url{www.digikey.com} bestellt und wurde innerhalb kurzer Zeit geliefert. Um die Platinen zu bestellen, mussten zunächst Gerberfiles erstellt werden, welche dann an Q-print electronic GmbH geschickt wurden.
Zum Bestücken der Platine war Teamarbeit gefragt, während einer das Bauteil möglichst exakt auf den kleinen Pads positioniert muss der Andere das erste Beinchen fest löten. Die wohl größte Herausforderung hierbei war das befestigen des FT321X auf der Kommunikationsplatine. Nachdem diese  gemeistert war, stellten die anderen Bauteile keine größeren Probleme mehr dar.
Als Änderung für neue Platinen wäre bei der Anzeigenplatine ein Größerer $RSET$ Widerstand einzubauen, da der eingebaute 10 $k\Omega$ etwas zu klein dimensioniert ist. Wenigstens 11,35 $k\Omega$ siehe Datenblatt AS1108 Seite 9.

\subsection{Softwareentwicklung}
Es musste nicht nur ein Treiber für den MSP geschrieben werden um über UART und SPI mit unseren Boards zu kommunizieren, sondern auch ein Treiber, welcher die Kommunikation zwischen PC und Kommunikationsplatine übernimmt. Die genauere beschreibung folgt im nächsten Kapitel.

