
 \begin{lstlisting}[style=PYTHON,frame=single,
  caption=Erstellen der Grafiken
  captionpos=b,
  label=lst:plot]
  import numpy as np
  import matplotlib.pyplot as plt
  
  # vergleich der Eingangsspannung zum mikrophon Signal
  def ein_ausgang():
      input_sig = np.genfromtxt('sig1000Hz.csv', delimiter=',')
      input_mic = np.genfromtxt('mic1000Hz.csv', delimiter=',')    
      
      plt.figure(figsize=(8, 5), dpi=800)
      plt.xticks(range(0,2501,250))
      plt.ylabel("Amplitude")
      plt.xlabel("Abtastzeitpunkt")
      resize_mic = input_mic[:,1:2] * 40  # mal 40, da mikrophonsignal etwa 40 mal schwaecher ist
      plt.plot(input_sig[:,1:2], color='y', label='1kHz Eingang')
      plt.plot(resize_mic, color='b', label='Ausgang')
      plt.plot((845, 845), (-1, 1), '-', color='red',label='Phasenverschiebung')
      plt.plot((940, 940), (-1, 1), '-', color='red')
      plt.legend()
      plt.show()
  
  def amplitude(datak, datag, x):
      datak = datak / 1500   # normierung
      datak = np.log10(np.abs(datak))
      datak = datak * 20    # dB
      datag = datag / 1500   # normierung
      datag = np.log10(np.abs(datag))
      datag = datag * 20    # dB
      plt.figure(figsize=(8, 5), dpi=800)   
      plt.semilogx(x, datak, label="kleiner Lautsprecher")
      plt.semilogx(x, datag, label="grosser Lautsprecher")
      plt.xlabel("Frequenz")
      plt.ylabel("Amplitudengang [$dB$]")
      plt.legend()
      plt.show()
  
  def phase(datak, datag, x):
      plt.figure(figsize=(8, 5), dpi=800)   
      plt.semilogx(x, datak, label="kleiner Lautsprecher")
      plt.semilogx(x, datag, label="grosser Lautsprecher")
      plt.xlabel("Frequenz")
      plt.ylabel("Phasengang")
      plt.legend()
      plt.show()
      
  # Beispielaufruf:
  # x = alle Messwerte, amplitude_gross = alle gemessenen Amplitudenwerte
  # zur jeweiligen frequenz, siehe Tabelle im Bericht
  
  #x = np.array([100,200,300,400,500,700,850,1000,1200,1500,1700,2000,3000,4000,5000,6000,10000])
  
  #amplitude_gross = np.array([50.2,134.0,95.2,60.4,49.2,
  #                   39.6,43.2,42.0,35.2,33.6,
  #                   30.8,32.0,37.6,46.0,26.8,
  #                   19.4,15.4])
  
  #amplitude_klein = np.array([6.6,10.4,15.0,23.8,34.6,
  #                   70.0,68.0,50.0,40.0,52.8,
  #                   53.6,51.2,50.4,52.0,12.8,
  #                   8.2,19.4])
                 
  #amplitude(amplitude_klein, amplitude_gross, x)
  
 \end{lstlisting}
 